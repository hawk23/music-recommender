% v2-acmsmall-sample.tex, dated March 6 2012
% This is a sample file for ACM small trim journals
%
% Compilation using 'acmsmall.cls' - version 1.3 (March 2012), Aptara Inc.
% (c) 2010 Association for Computing Machinery (ACM)
%
% Questions/Suggestions/Feedback should be addressed to => "acmtexsupport@aptaracorp.com".
% Users can also go through the FAQs available on the journal's submission webpage.
%
% Steps to compile: latex, bibtex, latex latex
%
% For tracking purposes => this is v1.3 - March 2012

\documentclass[prodmode,acmtecs]{acmsmall} % Aptara syntax

% Package to generate and customize Algorithm as per ACM style
\usepackage[ruled]{algorithm2e}
\renewcommand{\algorithmcfname}{ALGORITHM}
\SetAlFnt{\small}
\SetAlCapFnt{\small}
\SetAlCapNameFnt{\small}
\SetAlCapHSkip{0pt}
\IncMargin{-\parindent}

% Metadata Information
\acmVolume{0}
\acmNumber{0}
\acmArticle{1}
\acmYear{2015}
\acmMonth{10}

% Copyright
%\setcopyright{acmcopyright}
%\setcopyright{acmlicensed}
\setcopyright{rightsretained}
%\setcopyright{usgov}
%\setcopyright{usgovmixed}
%\setcopyright{cagov}
%\setcopyright{cagovmixed}

% DOI
%\doi{0000001.0000001}

%ISSN
%\issn{1234-56789}

% Document starts
\begin{document}

% Page heads
\markboth{}{Music Similarity and Retrieval: Content- and Context-based Approaches}

% Title portion
\title{Report on practical exercises for \emph{Music Similarity and Retrieval: Content- and Context-based Approaches}}
\author{
Verena Dittmer
\affil{Alpe Adria University of Klagenfurt}
Mario Graf
\affil{Alpe Adria University of Klagenfurt}
Peter Luca Lidl
\affil{Alpe Adria University of Klagenfurt}
}

\begin{abstract}
TODO: Abstract
\end{abstract}

%
% The code below should be generated by the tool at
% http://dl.acm.org/ccs.cfm
% Please copy and paste the code instead of the example below. 
%
\begin{CCSXML}
<ccs2012>
<concept>
<concept_id>10002951.10003317.10003347.10003352</concept_id>
<concept_desc>Information systems~Information extraction</concept_desc>
<concept_significance>300</concept_significance>
</concept>
</ccs2012>
\end{CCSXML}

\ccsdesc[300]{Information systems~Information extraction}

%
% End generated code
%

% We no longer use \terms command
%\terms{Design, Algorithms, Performance}

\keywords{TODO, TODO, TODO}

\begin{bottomstuff}
This is a lot of bottom stuff.
\end{bottomstuff}

\maketitle


\section{Introduction}

\subsection{Problem Formulation}

\section{Performance Evaluation}

\section{Conclusions}


% Start of "Sample References" section

\section{Typical references in new ACM Reference Format}
A paginated journal article \cite{Abril07}, an enumerated
journal article \cite{Cohen07}, a reference to an entire issue \cite{JCohen96}.

% Appendix
\appendix
\section*{APPENDIX}
\setcounter{section}{1}
TODO

% Bibliography
\bibliographystyle{ACM-Reference-Format-Journals}
\bibliography{msrcc-report-bibfile}
                             % Sample .bib file with references that match those in
                             % the 'Specifications Document (V1.5)' as well containing
                             % 'legacy' bibs and bibs with 'alternate codings'.
                             % Gerry Murray - March 2012


\medskip

\end{document}
% End of v2-acmsmall-sample.tex (March 2012) - Gerry Murray, ACM


